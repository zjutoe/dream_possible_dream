% !TEX encoding = UTF-8 Unicode
\documentclass[12pt,a4paper]{article}
\usepackage{fontspec,xunicode,xltxtra}
\usepackage{xeCJK} 
\usepackage{titlesec}
\usepackage[top=1in,bottom=1in,left=1.25in,right=1.25in]{geometry}

\titleformat{\section}{\Large\whei}{\thesection}{1em}{}

\XeTeXlinebreaklocale ``zh''
\XeTeXlinebreakskip = 0pt plus 1pt minus 0.1pt

% fonts in Mac OS
\newfontfamily\baoli{Baoli SC}
\newfontfamily\biaukai{BiauKai}
\newfontfamily\hei{Hei}
\newfontfamily\heiti{Heiti SC}
\newfontfamily\kai{Kai}
\newfontfamily\kaiti{Kaiti SC}
\newfontfamily\lantinghei{Lantinghei SC}
\newfontfamily\libian{Libian SC}
\newfontfamily\lihei{LiHei Pro}
\newfontfamily\lisong{LiSong Pro}
\newfontfamily\song{Songti SC}
\newfontfamily\fsong{STFangsong}
\newfontfamily\stheiti{STHeiti}
\newfontfamily\stkaiti{STKaiti}
\newfontfamily\stsong{STSong}
\newfontfamily\weibei{Weibei SC}
\newfontfamily\xingkai{Xingkai SC}
\newfontfamily\lisung{Apple LiSung}
\newfontfamily\whei{WenQuanYi Zen Hei}
\newfontfamily\wheio{WenQuanYi Zen Hei Mono}
\newfontfamily\wmhei{WenQuanYi Micro Hei}
\newfontfamily\wmheil{WenQuanYi Micro Hei Light}
\newfontfamily\wmheio{WenQuanYi Micro Hei Mono}
\newfontfamily\wmheilo{WenQuanYi Micro Hei Mono Light}

\setmainfont{STFangsong}

%% fonts in Windows
%%
%% \newfontfamily\bwei{FZBeiWeiKaiShu-S19S}
%% \newfontfamily\zbhei{FZZhanBiHei-M22T}
%% \newfontfamily\xzt{FZXiaoZhuanTi-S13T}
%% \newfontfamily\xbsong{FZXiaoBiaoSong-B05}
%% \newfontfamily\dbsong{FZDaBiaoSong-B06}
%% \newfontfamily\gulif{FZGuLi-S12T}
%% \newfontfamily\gulij{FZGuLi-S12S}
%% \newfontfamily\kai{Kai}
%% \newfontfamily\hei{FZHei-B01}
%% \newfontfamily\wsharp{WenQuanYi Zen Hei Sharp}
%% \newfontfamily\fsong{STFangsong}
%% \newfontfamily\stsong{STSong}
%% \newfontfamily\lanting{FZLanTingSong}
%% \newfontfamily\boya{FZBoYaSong}
%% \newfontfamily\lishu{FZLiShu-S01}
%% \newfontfamily\lishuII{FZLiShu II-S06}
%% \newfontfamily\yao{FZYaoTi-M06}
%% \newfontfamily\zyuan{FZZhunYuan-M02}
%% \newfontfamily\xhei{FZXiHei I-Z08}
%% \newfontfamily\xkai{FZXingKai-S04}
%% \newfontfamily\ssong{FZShuSong-Z01}
%% \newfontfamily\bsong{FZBaoSong-Z04}
%% \newfontfamily\nbsong{FZNew BaoSong-Z12}
%% \newfontfamily\caiyun{FZCaiYun-M09}
%% \newfontfamily\hanj{FZHanJian-R-GB}
%% \newfontfamily\songI{FZSongYi-Z13}
%% \newfontfamily\hcao{FZHuangCao-S09}
%% \newfontfamily\wbei{Weibei SC}
%% \newfontfamily\huali{FZHuaLi-M14}
%% \setmainfont{FZLanTingSong}

\renewcommand{\baselinestretch}{1.25}

\begin{document}

\title{\whei 追可行之梦想}
\author{\kai 拉吉·瑞迪\\
  \kai 叶敏娇~译}
\date{\kai ACM计算机科学大会~图灵奖演讲\\
  1995年3月1日}

\maketitle

\section{摘要}
本文对人工智能在计算机科学领域内和在全社会所扮演的角色做了若干回顾性与
展望性的评论。文中包括针对如下问题的回答:人工智能可以与人类相等吗?人
工智能仅仅是某种特殊的算法吗?人工智能不就是软件吗?为何全社会要支持人
工智能和计算机科学的研究?人工智能的下一步是什么?诸如此类。主题则是,
人工智能仍将是一个值得追寻、且能成真的可行之梦想。

\section{介绍}
很荣幸也很高兴获得ACM的这个奖项。尤为可喜的是此奖能与挚友兼30年老同事爱
德华·费根鲍姆(Ed Feigenbaum)共享。

作为第二代的人工智能研究者,我有幸认识多位人工智能之父并曾与他们共事。
在60年代,斯坦福人工智能实验室的黄金时期,我从导师约翰·麦卡锡(John
McCarthy)身上学到了鼓励研究的多样性、培育看似与自己方向不相关的研究领
域的重要性。虽然他的研究兴趣主要集中在常识性推理与认知论,但在他的领导
下,语音、视觉、机器人、语言、知识系统、游戏、音乐等研究都在人工智能实
验室发展兴旺。此外,许多开拓性的系统研究也在多个领域活跃而繁荣,比
如Lisp、分时系统、视频显示,还有一个叫做『玻璃片』的视窗系统先驱,等
等。

马文·闵斯基66年时正访问斯坦福,帮忙建火星车。从他那里,我学到了勇于追求
大胆的愿景。而我在卡内基梅隆大学二十多年的同事兼导师艾伦·纽厄尔(Allen
Newell)和赫伯特·西蒙(Herb Simon),则教会我如何将大胆的愿景通过仔细设
计、实验、遵循科学方法,最终变为现实。

我也十分幸运能结识艾伦·帕里斯并与之共事。他是五六十年代计算科学舞台上的
巨人,也是1966年在洛杉矶举行的ACM大会上第一届图灵奖得主。我也参加了那次
大会,彼时我还只是斯坦福的一名研究生。

我不认识艾伦·图灵,不过我可能是在座少数几位使用过他亲手设计的计算机的幸
运儿之一。50年代晚期,我有幸用过一台汞迟延线计算机(英国电气公司
的Deuce Mark II),它是基于图灵设计的ACE计算机。有鉴于图灵早年在智能机
器方面的论文,他不愧是人工智能奠基之父之一,与范内瓦·布什等先驱并列齐
名。

至此,我们就谈到了今天的主题,『追可行之梦』。许多人认为人工智能是虚幻
之梦,但我们身处人工智能领域却不以为然。它不仅是可行之梦,而且从某种角
度而言,人工智能已经是一个现实,四十年来已不断地展现了其成果。并且将来
可以预见的影响之大,将几十成百倍于迄今为止的成就。本次演讲中,我将揭开
人工智能研究者工作的神秘面纱,并探索人工智能的本质和它与算法、软件系统
研究之间的关系。我会探讨如今人工智能所能胜任的事情和它对社会的影响,最
后我会总结一下长远来看面临的巨大挑战。

\section{人类与其它形式的智能}

机器能显示出真正的智能吗?西蒙给出了一个深刻的答案:『我只知道一种操作
意义上的「智能」。如果人类完成了某种被视为有智能的工作,那么同样的工作
由一个或一系列(精神的)行为所完成时,我们可以说这个行为或这一系列行为
就是有智能的。我知道我的朋友有智能,因为他棋下得不错(能在路上开车、能
诊断疾病、能解传教士与食人族难题\footnote{一种过河算法问题——译者},诸如
此类)。我知道A计算机有智能因为它棋下得不错(在全球仅次于约200个人类)。
我知道Navlab\footnote{一个自动驾驶程序——译者}有智能因为它能在路上开车,
等等等等。那些认为计算机智能尚未到来的人,他们的问题就在于他们从来没有
认真研究过人类的智能。咱们要不试试写本「哪些事人类做不到?」,至少得跟
德雷福斯的书一样长。至迟从1956年,当LT\footnote{一个逻辑推理程序——译者}找
到一个比怀特海德和罗尔斯更好的证明起,或者当西屋电器的工程师们写了一个
能自动设计电子马达的程序起,计算机智能就已经是一个事实了。所以,我们谈
论计算机智能时,请不要再用将来时态了。』

计算机智能可以与人类相等吗?一些哲学家和物理学家\footnote{可能是暗讽
  《皇帝新脑》一书作者罗杰·彭罗斯(Roger Penrose)——译者}通过努力回答这
个问题实现了相当成功的终身事业。答案是人工智能既可以比人类智能多,也可
以比人类智能少。要证明这两者不能$100\%$相等其实无需长篇大论。人类智能有
些特性可能是人工智能系统不会有的(或者是因为我们没有特别的动机去实现,
或者是因为我们尚未开始考虑),反之,人工智能有些特性也是人类智能所无法
企及的。根本而言,人工智能可以做到什么,更多地是取决于社会需求以及人工
智能在哪些领域有『相对优势』,而非由哲学思考决定。

下面我通过两个类比来阐述这个论点。这是两个问题当前正是信息行业的研究热
点,即数字图书馆和电子商务。它们本身并非人工智能问题,但解决它们需要综
合一些人工智能技术。

数字图书馆的基本构成是电子书。电子书像实体书一样提供某些信息,人们可以
像阅读实体书一样阅读使用这些信息。躺床上看电子书还是比较困难,但可以预
见,技术进步之后,你可以躺床上读一种轻于12盎司、$6\times8$英寸高分辨率
彩屏、外观和感觉都像实体书的『亚笔记本』计算机。但是,它跟实体书的类比
也就仅止于此了。电子书不可能成为古籍善本供收藏之用,也不能用来生火,在
寒夜里给你带来温暖,你能用它来打架,但它估计挺贵的。但另一方面,用电子
书你可以处理、索引和搜索信息,直达正确的页面,高亮信息,如果你没戴眼镜
还能改变字体大小,等等。电子书跟实体书不一样,它既比后者多,也比后者
少。

电子商务的一个关键部分是电子商场。在这个虚拟商场里,你能步入一家店铺,
试穿一些『虚拟』衣物,欣赏自己的形象,下一个订单,然后坐等实物在24小时
内快递送到家里。显然,你无法体会到逛真正的商场那种兴奋的感觉,不能跟人
摩肩接蹱,也不能试穿真正的衣服。但是同样你也无需为了出门精心打扮,无需
与糟糕的交通作战,也无需排队等待。更重要的是,你足不出户,就能从巴黎买
衣服,从米兰买鞋子,从香港买劳力士表。同样,电子商场跟实体商场不一样,
它即比后者多,也比后者少。

与此相似,人工智能与人类智能相比较,既比后者多,也比后者少。人类总有一
些能力也许是人工智能拥有无法企及的。『能』与『不能』的界限将继续随时间
推移而不断改变。但更重要的是,人工智能会有一些超越人类的功能,从而拓展
了人类个体或群体的能力和所及之范围。拥有这些工具的人,将使其余人显得像
原始部落一样——顺便一提,我们人类创造的每种人造物其实都是如此,比如飞机。
人工智能只不过凑巧是一种增强人类的头脑心智能力的人造物而已。

\section{人工智能与算法}

人工智能不就是一类特殊的算法吗?某种意义上,是的;但它是一类极其丰富的
算法,迄今尚未得到足够的重视。其次,人工智能研究的一个主要部分并非仅关
心解决问题,而且同样还关心问题的定义。与复杂性理论学家类似,人工智能研
究者也倾向于关心NP完全问题,但不像他们对给定问题的复杂性感兴趣,人工智
能研究的焦点倾向于围绕着能提供近似、『意足』
(satisficing)\footnote{Satisficing是西蒙拼合satisfy(满足)
  和suffice(足够)生造的词,一般译为『满意度』,但实际含义是『未必最佳
  但已经令人满意』,所以译者认为截取成语『心满意足』的『意足』二字是个
  更精确的翻译。另一个可行而现成的字眼是『厌足』,可惜现代人一般不清楚
  厌字的原意『满意、满足』,只好割爱——译者}但不能保证最优解的算法。

『意足方案』的概念来自西蒙在组织决策理论上的开创性研究,他藉此获得了诺
贝尔奖。西蒙之前关于人类决策的研究认为,给定所有的相关事实,人类能够理
性地权衡这些因素。西蒙的研究表明,人类的思考受限于计算能力,导致人们对
『够好』的方案就已心满『意足』,并不强求权衡所有因素、得到最佳理性方案。
西蒙把这称为『有限理性』原则。当我们必须在超出思考能力的条件下作出决定
时,并不会说『这是个NP完全问题』然后放弃。我们会使用『最少计算量搜索』
策略,而非那些『最短路径搜索』策略。

『最少计算搜索』研究的是近似算法。在给定计算限制的条件下,比如有限的内
存、时间或带宽,近似算法要找到最佳解决方案。这是值得复杂性理论学家们认
真研究的领域。

除了解决指数级增长的问题之外,人工智能算法经常需要满足以下限制:具有面
向目标的适应性行为、从经验中学习、应用大规模知识、包容沟通错误与歧义的、
与人类用语言和语音交互、实时响应。

具有面向目标的适应性行为的算法。当问题的算法描述是『做什么』而非『如何
做』的形式时,目标和子目标就自然而然出现了。举例来说,考虑一个简单的任
务,叫一个智能体『帮我接通肯』。这要求将这个目标转换为几个子目标,比如
在电话本里查肯的号码,拨打号码,跟应答的智能体通话,等等。之后每个子任
务必须从『做什么』转换为『怎么做』,并执行之。创建与执行计划,已经在人
工智能里研究得很多了。另外的系统,比如报告生成、4GL(第四代语言)系统、
以及数据库『按例子查询』的方法用简单的基于模板的方案来解决『做什么到怎
么做转换』的问题。总的来说,为了解决这类问题,一个算法必须能够创建一个
包含所有目标(Goal)的计划,并用已知的操作(Operations)和方法
(Methods)来完成这些目标。这称为GOMS方法。在专家系统里,最常用的是一种
称为『手段目的分析』的面向目标的行为。

能从经验中学习的算法。从经验中学习,意味着算法有内建的机制来修改内部的
结构和功能。例如,在『帮我接通肯』的任务中,假设『肯』有歧义,你帮助智
能体找到了正确的那个肯,下次你叫智能体接通肯的时候,它应该使用与你相同
的启发式规则来解决歧义。这意味着智能体能够获得、表示并使用新知识,并在
必要时能在学习中参与澄清误会的对话。在一些计算机科学的圈子里,动态修改
内部结构和功能被认为很危险,因为有可能意外地覆盖别的结构。但是,在学习
的任务中只要问题结构本身允许这种形式的学习,修改概率、修改表或数据结构
的内容,已经成功地得到应用。纽厄尔等人开发的Soar架构,使用了基于规则的
系统架构,能够发现和添加新的规则(实际上是产生的规则),可能是迄今为止
最雄心勃勃的能通过经验学习来改进的程序了。

能用语言和语音与人类交互的算法。有效使用语音和语言作为人机交互界面的算
法将来必将起到越来越基础的作用,尤其是普通人都用计算机来解决日常问题的
时代。在前面那个『帮我拨通肯』的任务中,如果智能体不能用语言、语音或其
它自然的方法与人沟通以澄清误解,智能体就很难获得广泛应用。使用语言和语
音,就涉及到不仅能处理歧义和语法错误,而且能解析和解释词汇量巨大的自然
语言的算法。

能有效利用大规模知识的算法。大规模知识的处理不仅需要大量的内存,而且带
来一个难题,即如何选择正确的知识来应用到具体的任务和场景中。约翰·麦卡锡
很喜欢用的一个例子是,假设有人问了这个问题:『里根现在正站着呢还是坐
着?』,一个系统如果有巨大的数据库,就会系统地搜遍上T的数据,最后发现它
不知道答案。而人类在面对这个问题是,可能立刻就会说『我不知道』,而且可
能还会加一句『管他呢』。如何设计一个算法,使得它『知道』自己『不知道』
什么,眼下还是一个尚未解决的问题。

能包容沟通错误与歧义的算法。在人与人的沟通中,错误和歧义简直就是是生活
本身。沃伦·泰特尔曼在25年前开发了一个叫『DWIM』的接口,意即『照我说的做』
(Do What I Mean)。可惜这类设想在当时注定无足轻重,因为程序要高效运行,
而且要能及时做出来。如今,G级PC已近在眼前,我们应该重拾这些容错的概
念和算法。我们应该开发能够检测歧义(比如,包括『无效』在内的多种可行的
解释),并利用同时并行求值或澄清误解对话的方式解决歧义。

有实时要求的算法。许多系统已经设法解决此类问题,但仅能通过小心而痛苦的
代码分析进行。很少有人研究过如何创建可以接受『加油!快点!』这类命令的
算法。Prodigy系统采用了针对这类问题的一种解决方案。它会立刻生成一个方案,
并且当系统发现有时间思考时,会逐渐改进方案以代替现有方案。因此,它随时
都能给出答案,但是慢慢地答案的质量会逐渐提高。有意思的是,数值分析中的
许多迭代算法可以改造为这种形式的『随时即有的答案』。

『自知』(self-aware)的算法。具有如下特点的算法可以笼统地称为具备『自
知』机制:能解释自己能力(例如能回答『如何做』和『假如如何如何』之类问
题的帮助命令),监控、诊断、感染病毒时修复自己。在线超文本手册和校验码
算是这类自知机制的较简单的例子。回答『如何做』和『假如如何如何』之类问
题比较困难。监控和诊断病毒意味着『解析』来访的请求,而非简单地执行之。

这些算法的设计问题从人工智能领域自然地出现,并且研究人员已经创造了特定
情境下的许多解决方案。这些解决方案的进一步发展和概括将会丰富整个计算机
科学。

\section{软件系统与人工智能}

人工智能不就是软件吗?电视机不就是电器吗?原则上而言,人工智能确实就是
软件而已,虽然人工智能系统一般比较大且复杂。要构建人工智能系统通常意味
着要构建必要的软件工具。比如,编程语言的许多早期进步都要归功于人工智能
研究人员,例如链表结构、指针、虚拟内存、动态内存分配、内存垃圾回收等
等。

庞大的人工智能系统,尤其是那些部署了且交付日常应用的系统,存在其它复杂
软件系统也有的问题,例如『延期交付』、『超支』、『不稳定』等等。『人月
神话』\footnote{译者按:参见《The Mythical Man Month:Essays on
  Software Engineering》,作者是IBM System/360系统之父弗雷德·布鲁克
  斯}原则同样会惩罚人工智能系统。这些系统不仅大而复杂,而且经常不能参考
以前设计的类似的系统。所以,跟传统的软件工程任务一样,很难形成需求说明
或测试流程。

在人工智能的情境下,涌现了许多新概念,可以給整个计算机科学启发一些新的
路线和方法:

即插即用架构。做一个综合的人工智能系统,具备面向目标的适应性行为、从经
验中学习、应用大规模知识、包容沟通错误与歧义的、与人类用语言和语音交互、
实时响应等等特性的组建,我们不能每次都从头做起。在这类系统中,应用相关
的组建可能还不足整体工作量的$10\%$。如果系统需要在合理的时间内就运转起
来,我们需要依靠接口和由『即插即用』组建构成的软件架构。这个想法与黑
盒\footnote{原文作blackboard,疑误} 概念相似:协作的智能体能一起工作,
但无需显式地相互引用。

80/20原则。过去40年构建显示智能行为的系统的努力,证明了实际上这个任务是
如何艰巨。最近的一种模式转变是,放弃完全替代人类的自足的系统,转向辅助
人类的智能体。例如,与其创造一个翻译系统,还不如建一个『翻译助手』,能
够給人类译者提供最佳译文选择。于是,目标就变成了建一个系统做$80\%$的工
作,剩下的$20\%$交给人类用户。这样的系统运转之后,研究工作可以针对向剩
下的$20\%$,在下一版中重复上述80/20原则。于是这个系统就逐步靠近人类的能
力,同时已知提供一个可用的工具,每次版本升级也许就提升3至5成的人类工作
效率。不过这个任务不像看起来那么简单。在这种新的模式中,老问题『一个系
统如何知道它不知道什么东西』又冒出来了。要一个智能体有能力说『稍等,我
叫我上级来』,它必须是自知的!它必须知道自己能做什么,不能做什么。这也
不是个简单的任务,但无论如何得解决。

快速失败策略。在工程设计领域又两条广泛接受的实践经验:『第一次就做对』
和『快速失败』。前者用于设计之前已经生产多次的产品。而构建之前从未建过
的系统或机器人时,想要『第一次就做对』恐怕不是最佳策略。最近我们
給NASA(美国宇航局)建造用于探索火山环境的机器人『但丁』
(Dante)\footnote{译者按:可能是用了但丁名著《神曲》炼狱的典故。} 的一
个实验就是个例子。绝大多数NASA实验都要求不能失败,因为许多任务都是人命
关天。要求毫无错误,导致15年之久的计划周期和十亿美元的预算。快速失败策
略是说,如果你要建造一个以前从来没有造过的复杂系统,最好就是建造一系列
抛弃原型,改进学习曲线。因为这类系统会以意料之外的方式失败,所以我们应
该将失败视为通向成功的阶梯,而非不可接受的结果。我们第一个但丁走了几步
就失败了。第二个活了一周。现在我们正在建第三代的模型!与传统的任务相比,
这个实验所费时间和预算都都至少低一个数量级。

科学的方法。在一次NRC(美国国家研究委员会)的研讨会上,一位物理学家问道:
『计算机科学的所谓科学在哪里?』现在我可以很高兴地说了,在一些人工智能
系统里,我们已经有『假设、实验、证实和重复』研究范式的例子了。过去10年
中,大部分语音识别研究已经遵循了这条道路。利用共享的数据库,相互竞争的
模型利用可运行的系统加以评估。于是在别的系统里,成功的想法数月之内就神
奇地冒出来的,导致一些语音建模的机制被证实或拒绝。ARPA(美国国防部高级
研究计划局)在要求科研圈子使用这种证实过程方面居功厥伟。所有实验性的计
算机科学应该都能从这种严格的实验中受益。正如纽厄尔曾经说过,不要争论,
我们可以设计实验来证明或证伪这个想法!

\section{结论}

我的总结就是,人工智能仍旧是一个值得追寻的可行的梦想。人工智能的进步非
常显著。人工智能将继续产生有趣的问题和解决方案。尽管如此,人工智能的目
标最终将在更广阔的计算机科学大环境中得到实现,比如G级PC,即每秒十亿次操
作而价格仅与现在的PC相当的计算机,软件工具、环境和算法的进步,等等。

\section{致谢}

我很幸运能得到ARPA近30年的资助。在此我要感谢Lick Licklider,Ivan
Sutherland,Larry Roberts和Bob Kahn,虽然人工智能并非他们个人的研究范围,
但他们仍然以长远的眼光赞助人工智能研究。我也要感谢ARPA现在的领导
们:Duane Adams, Ed Thompson and Allen Sears,他们延续了这一传统。

我很感激赫伯特·西蒙、艾德·费根鲍姆、詹米·卡伯耐尔(Jaime Carbonell)、
汤姆·米歇尔(Tom Mitchell)、杰克·莫斯托(Jack Mostow)、查克·索普
(Chuck Thorpe)以及吉姆·克歇尔(Jim Kocher),他们在我准备这篇论文时给
了许多建议、评论和帮助。


\end{document}
