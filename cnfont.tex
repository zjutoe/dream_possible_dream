% !TEX encoding = UTF-8 Unicode
\documentclass[12pt,a4paper]{article}
\usepackage{fontspec,xunicode,xltxtra}
\usepackage{xeCJK} 
\usepackage{titlesec}
\usepackage[top=1in,bottom=1in,left=1.25in,right=1.25in]{geometry}

\titleformat{\section}{\Large\whei}{\thesection}{1em}{}

\XeTeXlinebreaklocale ``zh''
\XeTeXlinebreakskip = 0pt plus 1pt minus 0.1pt

% fonts in Mac OS
\newfontfamily\baoli{Baoli SC}
\newfontfamily\biaukai{BiauKai}
\newfontfamily\hei{Hei}
\newfontfamily\heiti{Heiti SC}
\newfontfamily\kai{Kai}
\newfontfamily\kaiti{Kaiti SC}
\newfontfamily\lantinghei{Lantinghei SC}
\newfontfamily\libian{Libian SC}
\newfontfamily\lihei{LiHei Pro}
\newfontfamily\lisong{LiSong Pro}
\newfontfamily\song{Songti SC}
\newfontfamily\fsong{STFangsong}
\newfontfamily\stheiti{STHeiti}
\newfontfamily\stkaiti{STKaiti}
\newfontfamily\stsong{STSong}
\newfontfamily\weibei{Weibei SC}
\newfontfamily\xingkai{Xingkai SC}
\newfontfamily\lisung{Apple LiSung}
\newfontfamily\whei{WenQuanYi Zen Hei}
\newfontfamily\wheio{WenQuanYi Zen Hei Mono}
\newfontfamily\wmhei{WenQuanYi Micro Hei}
\newfontfamily\wmheil{WenQuanYi Micro Hei Light}
\newfontfamily\wmheio{WenQuanYi Micro Hei Mono}
\newfontfamily\wmheilo{WenQuanYi Micro Hei Mono Light}

\setmainfont{STFangsong}

%% fonts in Windows
%%
%% \newfontfamily\bwei{FZBeiWeiKaiShu-S19S}
%% \newfontfamily\zbhei{FZZhanBiHei-M22T}
%% \newfontfamily\xzt{FZXiaoZhuanTi-S13T}
%% \newfontfamily\xbsong{FZXiaoBiaoSong-B05}
%% \newfontfamily\dbsong{FZDaBiaoSong-B06}
%% \newfontfamily\gulif{FZGuLi-S12T}
%% \newfontfamily\gulij{FZGuLi-S12S}
%% \newfontfamily\kai{Kai}
%% \newfontfamily\hei{FZHei-B01}
%% \newfontfamily\wsharp{WenQuanYi Zen Hei Sharp}
%% \newfontfamily\fsong{STFangsong}
%% \newfontfamily\stsong{STSong}
%% \newfontfamily\lanting{FZLanTingSong}
%% \newfontfamily\boya{FZBoYaSong}
%% \newfontfamily\lishu{FZLiShu-S01}
%% \newfontfamily\lishuII{FZLiShu II-S06}
%% \newfontfamily\yao{FZYaoTi-M06}
%% \newfontfamily\zyuan{FZZhunYuan-M02}
%% \newfontfamily\xhei{FZXiHei I-Z08}
%% \newfontfamily\xkai{FZXingKai-S04}
%% \newfontfamily\ssong{FZShuSong-Z01}
%% \newfontfamily\bsong{FZBaoSong-Z04}
%% \newfontfamily\nbsong{FZNew BaoSong-Z12}
%% \newfontfamily\caiyun{FZCaiYun-M09}
%% \newfontfamily\hanj{FZHanJian-R-GB}
%% \newfontfamily\songI{FZSongYi-Z13}
%% \newfontfamily\hcao{FZHuangCao-S09}
%% \newfontfamily\wbei{Weibei SC}
%% \newfontfamily\huali{FZHuaLi-M14}
%% \setmainfont{FZLanTingSong}

\renewcommand{\baselinestretch}{1.25}

\begin{document}

\title{\whei 梦可行之梦想}
\author{\kai 拉吉·瑞迪\\
  \kai 叶敏娇~译}
\date{\kai 1995年3月1日}

\maketitle

\section{摘要}
本文汇集了对人工智能在计算机科学领域和全社会所扮演角色的若干回顾性与展
望性的评论。文中包括针对如下问题的评论:人工智能可以与人类相等吗?人工
智能仅仅是某种特殊的算法吗?人工智能不就是软件吗?为何我们社会要支持人
工智能和计算机科学的研究?人工智能的下一步是什么?诸如此类。主题则是,
人工智能仍将是一个值得追寻、且能成真的可行之梦想。

\section{介绍}
很荣幸也很高兴获得ACM的这个奖。尤其可喜的是与挚友兼30年同事爱德华·费根
鲍姆(Ed Feigenbaum)共享这一奖项。

作为第二代的人工智能研究者,我很幸运认识多位人工智能之父并曾与他们共事。
在60年代人工智能实验室的黄金时期,我从导师约翰·麦卡锡(John McCarthy)
身上学到了鼓励研究的多样性、培育看似与自己方向不相关的研究领域的重要性。
虽然他的研究兴趣主要集中在常识性推理与认知论,但在他的领导下,语音、视
觉、机器人、语言、知识系统、游戏、音乐等研究都在人工智能实验室发展兴旺。
此外,许多开拓性的系统研究也在不同领域活跃而繁荣,比如Lisp、分时系统、
视频显示,还有一个叫做『玻璃片』的视窗系统先驱,等等。

马文·闵斯基在66年访问斯坦福,并帮着建了火星车。从他那里,我学到了追求大
胆愿景的重要性。而我卡内基梅隆大学二十多年的同事兼导师艾伦·纽厄尔
(Allen Newell)和赫伯特·西蒙(Herb Simon),则教会我如何将大胆的愿景通
过仔细设计、实验、遵循科学方法,最终变为现实。

我也十分幸运能结识艾伦·帕里斯并与之共事。他是五六十年代计算科学舞台上的
巨人,也是1966年在洛杉矶举行的ACM大会上第一届图灵奖得主。我也参加了那次
大会,彼时我还只是斯坦福的一名研究生。

我不认识艾伦·图灵,不过我可能是少数几位使用过他亲手设计的计算机的幸运儿
之一。50年代晚期,我有幸用过一台汞迟延线计算机(英国电气公司的Deuce
Mark II),它是基于图灵设计的ACE计算机。有鉴于图灵早年在智能机器方面的
论文,他不愧是人工智能奠基之父之一,与范内瓦·布什等先驱并列齐名。

至此,我们就谈到了今天的主题,『梦可行之梦』。许多人认为人工智能是虚幻
之梦,但我们不以为然。它不仅是可行之梦,而且从某种角度而言,人工智能已
经是一个现实,四十年来已不断地展现了其成果。并且未来可以预见的影响之大,
将几十成百倍于迄今为止的进步。本次演讲中,我会试着揭开人工智能研究者工
作的神秘面纱,并探索人工智能的本质和它与算法、软件系统研究之间的关系。
我会探讨如今人工智能所能胜任的事情和它对社会的影响,最后我会总结一下长
远来看面临的巨大挑战。

\section{人类与其它形式的智能}

机器能显示出真正的智能吗?西蒙给出了一个深刻的答案:『我只知道一种操作
意义上的「智能」。如果人类完成了某种被视为有智能的工作,那么同样的工作
由一个或一系列(精神的)行为所完成时,我们可以说这个行为或这一系列行为
就是有智能的。我知道我的朋友有智能,因为他棋下得不错(能在路上开车、能
诊断疾病、能解传教士与食人族难题\footnote{译者按:一种过河算法问题},诸
如此类)。我知道A计算机有智能因为它棋下得不错(在全球仅次于约200个人
类)。我知道Navlab\footnote{译者按:一个自动驾驶程序}有智能因为它能在路
上开车,等等等等。那些认为计算机智能尚未到来的人,他们的问题就在于他们
从来没有认真研究过人类的智能。咱们要不试试写本「哪些事人类做不到?」,
至少得跟德雷福斯的书一样长。至迟从1956年,当LT\footnote{译者按:一个逻
  辑推理程序}找到一个比怀特海德和罗尔斯更好的证明起,或者当西屋电器的工
程师们写了一个能自动设计电子马达的程序起,计算机智能就已经是一个事实了。
所以,我们谈论计算机智能时,请不要再用将来时态了。』

计算机智能可以与人类相等吗?一些哲学家和物理学家\footnote{译者按:可能
  是暗讽《皇帝新脑》一书作者}通过努力回答这个问题实现了相当成功的终身事
业。答案是人工智能既可以比人类智能多,也可以比人类智能少。要证明这两者
不能$100\%$相等其实无需长篇大论。人类智能有些特性可能是人工智能系统不会
有的(或者是因为我们没有特别的动机去实现,或者是因为我们尚未开始考虑),
反之,人工智能有些特性也是人类智能所无法企及的。根本而言,人工智能可以
做到什么,更多地是取决于社会需求以及人工智能在哪些领域有『相对优势』,
而非由哲学思考决定。

下面我通过两个类比来阐述这个论点。这是两个问题当前正是信息行业的研究热
点,即数字图书馆和电子商务。它们本身并非人工智能问题,但解决它们需要综
合一些人工智能技术。

数字图书馆的基本构成是电子书。电子书像实体书一样提供某些信息,人们可以
像阅读实体书一样阅读使用这些信息。躺床上看电子书还是比较困难,但可以预
见,技术进步之后,你可以躺床上读一种轻于12盎司、$6\times8$英寸高分辨率
彩屏、外观和感觉都像实体书的『亚笔记本』计算机。但是,它跟实体书的类比
也就仅止于此了。电子书不可能成为古籍善本供收藏之用,也不能用来生火,在
寒夜里给你带来温暖,你能用它来打架,但它估计挺贵的。但另一方面,用电子
书你可以处理、索引和搜索信息,直达正确的页面,高亮信息,如果你没戴眼镜
还能改变字体大小,等等。电子书跟实体书不一样,它既比后者多,也比后者
少。

电子商务的一个关键部分是电子商场。在这个虚拟商场里,你能步入一家店铺,
试穿一些『虚拟』衣物,欣赏自己的形象,下一个订单,然后坐等实物在24小时
内快递送到家里。显然,你无法体会到逛真正的商场那种兴奋的感觉,不能跟人
摩肩接蹱,也不能试穿真正的衣服。但是同样你也无需为了出门精心打扮,无需
与糟糕的交通作战,也无需排队等待。更重要的是,你足不出户,就能从巴黎买
衣服,从米兰买鞋子,从香港买劳力士表。同样,电子商场跟实体商场不一样,
它即比后者多,也比后者少。

与此相似,人工智能与人类智能相比较,既比后者多,也比后者少。人类总有一
些能力也许是人工智能拥有无法企及的。『能』与『不能』的界限将继续随时间
推移而不断改变。但更重要的是,人工智能会有一些超越人类的功能,从而拓展
了人类个体或群体的能力和所及之范围。拥有这些工具的人,将使其余人显得像
原始部落一样——顺便一提,我们人类创造的每种人造物其实都是如此,比如飞机。
人工智能只不过凑巧是一种增强人类的头脑心智能力的人造物而已。

\section{人工智能与算法}

人工智能不就是一类特殊的算法吗?某种意义上,是的;但它是一类极其丰富的
算法,迄今尚未得到足够的重视。其次,人工智能研究的一个主要部分并非仅关
心解决问题,而且同样还关心问题的定义。与复杂性理论学家类似,人工智能研
究者也倾向于关心NP完全问题,但不像他们对给定问题的复杂性感兴趣,人工智
能研究的焦点倾向于围绕着能提供近似、『意足』但不能保证最优解的算法。

『意足方案』\footnote{satisficing solution,satisficing是西蒙拼
  合satisfy和suffice生造的词,一般译为『满意度』,但译者认为『意足』是
  个更好的翻译。另一个可行的字眼是『厌足』,但现代人一般不清楚厌字的原
  意『满意、满足』,所以不用。}的概念来自西蒙在组织决策理论上的开创性研
究,他藉此获得了诺贝尔奖。西蒙之前的人类决策研究认为,给定所有的相关事
实,人类能够理性地权衡这些事实因素,作出正确的选择。西蒙的研究表明,人
类的思考受限于计算能力局限,导致人们觉得『够好』的方案就已心满『意足』,
并不强求权衡所有因素、得到理性最佳方案。西蒙把这称为『有限理性』原则。
当人们必须在超出思考能力的条件下作出决定时,我们并不放弃,说『这是
个NP完全问题』。我们会使用『最少计算搜索』策略,而非那些『最短路径搜索』
策略。

\begin{table}[htbp]
\caption{字体列表}

\centering
\begin{tabular}{|l|c|r|}
\hline
\hei 字体 & \hei 命令 & \hei 字体效果 \\
\hline
\kai 报隶 & \verb+\baoli+ & \baoli 那只狐狸跳过了那只懒狗 The quick brown fox \\
\kai 标楷 & \verb+\biaukai+ & \biaukai 那只狐狸跳过了那只懒狗 The quick brown fox \\
\kai 黑 & \verb+\hei+ & \hei 那只狐狸跳过了那只懒狗 The quick brown fox \\
\kai 黑体 & \verb+\heiti+ & \heiti 那只狐狸跳过了那只懒狗 The quick brown fox \\
\kai 楷 & \verb+\kai+ & \kai 那只狐狸跳过了那只懒狗 The quick brown fox \\
\kai 楷体 & \verb+\kaiti+ & \kaiti 那只狐狸跳过了那只懒狗 The quick brown fox \\
\kai 兰亭黑体 & \verb+\lantinghei+ & \lantinghei 那只狐狸跳过了那只懒狗 The quick brown fox \\
\kai 隶变 & \verb+\libian+ & \libian 那只狐狸跳过了那只懒狗 The quick brown fox \\
\kai 俪黑 & \verb+\lihei+ & \lihei 那只狐狸跳过了那只懒狗 The quick brown fox \\
\kai 俪宋 & \verb+\lisong+ & \lisong 那只狐狸跳过了那只懒狗 The quick brown fox \\
\kai 宋 & \verb+\song+ & \song 那只狐狸跳过了那只懒狗 The quick brown fox \\
\kai 仿宋 & \verb+\fsong+ & \fsong 那只狐狸跳过了那只懒狗 The quick brown fox \\
\kai 华文黑体 & \verb+\stheiti+ & \stheiti 那只狐狸跳过了那只懒狗 The quick brown fox \\
\kai 华文楷体 & \verb+\stkaiti+ & \stkaiti 那只狐狸跳过了那只懒狗 The quick brown fox \\
\kai 华文宋 & \verb+\stsong+ & \stsong 那只狐狸跳过了那只懒狗 The quick brown fox \\
\kai 魏碑 & \verb+\weibei+ & \weibei 那只狐狸跳过了那只懒狗 The quick brown fox \\
\kai 行楷 & \verb+\xingkai+ & \xingkai 那只狐狸跳过了那只懒狗 The quick brown fox \\
\kai 苹果俪宋 & \verb+\lisung+ & \lisung 那只狐狸跳过了那只懒狗 The quick brown fox \\
\kai 宋 & \verb+\song+ & \song 那只狐狸跳过了那只懒狗 The quick brown fox \\
\kai 楷 & \verb+\kai+ & \kai 那只狐狸跳过了那只懒狗 The quick brown fox \\
\kai 仿宋 & \verb+\fsong+ & \fsong 那只狐狸跳过了那只懒狗 The quick brown fox \\
\kai 文泉驿正黑 & \verb+\whei+ & \whei 那只狐狸跳过了那只懒狗 The quick brown fox \\
\kai 文泉驿正黑Mono & \verb+\wheio+ & \wheio 那只狐狸跳过了那只懒狗 The quick brown fox \\
\kai 文泉驿细黑 & \verb+\wmhei+ & \wmhei 那只狐狸跳过了那只懒狗 The quick brown fox \\
\kai 文泉驿细黑小 & \verb+\wmheil+ & \wmheil 那只狐狸跳过了那只懒狗 The quick brown fox \\
\kai 文泉驿细黑小Mono & \verb+\wmheilo+ & \wmheilo 那只狐狸跳过了那只懒狗 The quick brown fox \\
\kai 文泉驿细黑小Mono & \verb+\wmheio+ & \wmheio 那只狐狸跳过了那只懒狗 The quick brown fox \\
\hline

%% for windows
%
% \kai 宋体 & \verb+\song+ & \song 宋体 \\
% \kai 楷体 & \verb+\kai+ & \kai 楷体 \\
% \kai 黑体 & \verb+\hei+ & \hei 黑体 \\
% \kai 仿宋体 & \verb+\fsong+ & \fsong 仿宋体 \\
% \kai 文泉驿黑体 & \verb+\whei+ & \whei 文泉驿黑体 \\
% \kai 书宋体 & \verb+\ssong+ & \ssong 书宋体 \\
% \kai 报宋体 & \verb+\bsong+ & \bsong 报宋体 \\
% \kai 新报宋体 & \verb+\nbsong+ & \nbsong 新报宋体 \\
% \kai 兰亭宋体 & \verb+\lanting+ & \lanting 兰亭宋体 \\
% \kai 博雅宋体 & \verb+\boya+ & \boya 博雅宋体 \\
% \kai 宋体一 & \verb+\songI+ & \songI 宋体一 \\
% \kai 隶书 & \verb+\lishu+ & \lishu 隶书 \\
% \kai 隶书二 & \verb+\lishuII+ & \lishuII 隶书二 \\
% \kai 古隶简体 & \verb+\gulij+ & \gulij 古隶简体 \\
% \kai 古隶繁体 & \verb+\gulif+ & \gulif 古隶繁体 \\
% \kai 华隶书 & \verb+\huali+ & \huali 华隶书 \\
% \kai 小标宋 & \verb+\xbsong+ & \xbsong 小标宋 \\
% \kai 大标宋 & \verb+\dbsong+ & \dbsong 大标宋 \\
% \kai 小篆体 & \verb+\xzt+ & \xzt 小篆体 \\
% \kai 姚体 & \verb+\yao+ & \yao 姚体 \\
% \kai 准圆 & \verb+\zyuan+ & \zyuan 准圆 \\
% \kai 细黑一 & \verb+\xhei+ & \xhei 细黑一 \\
% \kai 行楷书 & \verb+\xkai+ & \xkai 行楷书 \\
% \kai 彩云体 & \verb+\caiyun+ & \caiyun 彩云体 \\
% \kai 汉简书 & \verb+\hanj+ & \hanj 汉简书 \\
% \kai 魏碑体 & \verb+\wbei+ & \wbei 魏碑体 \\
% \hline

\end{tabular}
\end{table}

\end{document}
